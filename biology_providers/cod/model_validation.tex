\documentclass[dvips,a4,graphicx]{article}          %% uncomment if embedded
\title{Validation of Cod model implementation}    %% uncomment if embedded
\author{A. Christensen(asch@aqua.dtu.dk)}           %% uncomment if embedded 
\begin{document}                                    %% uncomment if embedded
\maketitle                                          %% uncomment if embedded
\section{Validation of Cod model implementation}   
\subsection{Egg stage}   
\begin{itemize}
  \item Ref \cite{Daewel2011} Fig 3a:
\end{itemize}
\subsection{Non-feeding larvae}      
\subsection{Feeding larvae}    
\begin{itemize}
  \item Ref \cite{Daewel2011} Fig 3b: 
  Figure reproduced reasonable
  Premise in  Fig 3b is zbiomass = 100 mgC/m3 = 100/1e6/0.32 kgDW/m3 = 0.00031 kgDW/m3.
  This is implemented by {\tt local\_env%zbiomass = 100./1e6/0.32 ! kgDW/m3 converted from mgC/m3}
  using {\tt call test\_foraging\_window\_llarv(ref\_larv,local\_env,7.0,20.0,0.1)
  Capture success (CS) and handling time (HT) is only depends on predator/prey size ratio
  and therefor optimal foraging window only depends on larval size and prey size spectrum
  which depends on seasonal slope, but not directly on temperature. 
  Prey size spectrum representation is different in 
  current implementation (continuous spectrum + seasonal slope) and Ref \cite{Daewel2011}.
  Temperature independce of foraging window was also confirmed as expected.
 
  \item Ref \cite{Daewel2011} Fig 5a:

  Figure reproduced reasonable for temperature T = 4 $\circ$C and T = 12 $\circ$C
  however, growth tends to be a somewhat smaller than Ref \cite{Daewel2011} Fig 5a.
  The reason has not been identified, but the deviation is within the range that
  can be captured by minor parameter tweeking, e.g. reducing SDA to 0.3 instead of 0.35
  or increasing AE by 10 \%. We note that the growth equation eq 1 in  Ref \cite{Daewel2011}
  is different from Ref \cite{Lough2005}, eventhough parameters are imported directly from this 
  paper. We use the growth equation from Ref \cite{Lough2005} in this implementation
%  using zoo plankton density 150 mgC/m3 (oceanography provider {\tt zooplankton\_test/ztest\_const.f}

\end{itemize}
%% ==============================================================
\begin{thebibliography}{99}
%% ==============================================================
\bibitem{Daewel2008}
Daewel Ute; Peck Myron A.; Kuehn Wilfried; et al:
Coupling ecosystem and individual-based models to simulate the influence 
of environmental variability on potential growth and survival of larval sprat 
(Sprattus sprattus L.) in the North Sea
\newblock FISHERIES OCEANOGRAPHY \textbf{17}(5), 333--351 (2008)
%% --------------------------------------------------------------
\bibitem{Daewel2011}
Daewel Ute; Peck Myron A.; Schrum Corinna:
Life history strategy and impacts of environmental variability on early 
life stages of two marine fishes in the North Sea: an individual-based 
modelling approach
\newblock CANADIAN JOURNAL OF FISHERIES AND AQUATIC SCIENCES \textbf{68}(3), 426--443 (2011)
%% --------------------------------------------------------------
\bibitem{Lough2005}
Lough RG; Buckley LJ; Werner FE; et al.
A general biophysical model of larval cod (Gadus morhua) growth applied to populations on Georges Bank
\newblock FISHERIES OCEANOGRAPHY  \textbf{14}(4), 241--262 (2005)
%% --------------------------------------------------------------
\end{thebibliography}

  
\end{document}                                      %% uncomment if embedded
