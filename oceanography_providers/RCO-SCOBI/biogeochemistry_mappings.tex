\documentstyle{article}
\begin{document}

\begin{itemize}
   \item Redfield molar stoichiometry :   C:N:P == 106:16:1 \newline
         Redfield weight ratio        :   C:N:P == 41:7:1   \newline
         Ref: Redfield A.C., On the proportions of organic derivations in sea water and their relation to the composition of plankton.
         In James Johnstone Memorial Volume. (ed. R.J. Daniel). University Press of Liverpool, pp. 176–192, 1934.
         Redfield, A.C., The biological control of chemical factors in the environment, American Scientist, 1958
   \item Zooplankton DW/WW mass $\sim$ 0.19 \newline
         Ref: Makoto Omori, Marine Biology, May 1969, Volume 3, Issue 1, pp 4–10
         "33 species of zooplankton distributed predominantly in the open sea region of the North Pacific. 
         Sampling covered the waters from 44N to the equator. Average percentage of dry weight to wet weight was about 19 \% of all samples from the whole area"
   \item Zooplankton carbon weight content $\sim$ 0.4 x DW \newline
         Ref: J.A. Lindley D.B. Robins R. Williams ,Journal of Plankton Research Vol.21 no.11 pp.2053–2066,  1999
   \item Birds: CC(g) = 0.1807 x FW(g) = 0.4571 x DW(g)  \newline 
         Ref: Journal of Experimental Marine Biology and Ecology 481:41-48, 2016
   \item For chlorophyll, the conversion factor is 2 mg-Chl = mmol-N (Neumann 2000). 
   \item Cell proxy relation: 1 mg-N $\sim$ 1/8 mg-CHL\newline 
         Ref:Yentsch, Charles S.; Vaccaro, Ralph F., Phytoplankton Nitrogen in the Oceans, Limnology and Oceanography, 
         1958, Volume 3, Issue 4, pp. 443-448\newline 
         1mmol-N = 14 mg-N $\sim$ 14/8 mg-CHL $\sim$ 2 mg-CHL
   \item Converting ml/l to mmol/m3: 1 ml/l = 44.65948 mmol/m3\newline
         \begin{verbatim}
real, parameter :: O2_g_per_L   =  1.42905                       ! USGS, Office of Water Quality Technical Memorandum 2011.03
real, parameter :: O2_g_per_mol = 15.9994*2                      ! standard atomic weight
real, parameter :: mll_2_mmolm3 = 1.e3 * O2_g_per_L/O2_g_per_mol ! derived factor for converting ml/l    -> mmol/m3 
         \end{verbatim}
         Ref: Office of Water Quality Technical Memorandum 2011.03, Change to Solubility Equations for Oxygen in Water
\end{itemize}

\begin{tabular}{|c|c|c|c|c|} \hline
interpolate\_X & RCO-SCOBI       & RCO-SCOBI                &  IBMlib      & conversion factor \\ 
               & unit            &  parameter key \#        &  unit        & RCO-SCOBI $\rightarrow$ IBMlib) \\ \hline \hline
zooplankton    & mgC/m3          & 10                       &  kg DW/m3    & 1e-6/0.4  \\ \hline
diatoms        & mgCHL/m3        & 11                       &  mmol/m3     & 0.5       \\ \hline   
flagellates    & mgCHL/m3        & 12                       &  mmol/m3     & 0.5       \\ \hline    
cyanobacteria  & mgCHL/m3        & 13                       &  mmol/m3     & 0.5       \\ \hline    
organic\_detritus & mgC/m3       & 14                       &  mmolN/m3    & 12*16/106 \\ \hline 
ammonimum      & mmolN/m3        & 15                       &  mmol/m3     & 1         \\ \hline
nitrate        & mmolN/m3        & 16                       &  mmol/m3     & 1         \\ \hline     
phosphate      & mmolP/m3        & 17                       &  mmol/m3     & 1         \\ \hline        
oxygen         & ml/l            & 18                       &  mmol/m3     & 1e3*1.42905/15.9994/2   \\ \hline
POM            & NA              & NA                       &  mmolN/m3     & NA     \\ \hline   
DIC            & NA              & NA                       &  mmolN/m3     & NA     \\ \hline   
alkalinity     & NA              & NA                       &  mmolN/m3     & NA     \\ \hline   
DIN            & NA              & NA                       &  mmolN/m3     & NA     \\ \hline   
chlorophyl     & NA              & NA                       &  mgCHL/m3    & 1*(\#11+\#12+\#13) \\ \hline   
\end{tabular}

\end{document}

-------------------------------------------------------------------------------------------